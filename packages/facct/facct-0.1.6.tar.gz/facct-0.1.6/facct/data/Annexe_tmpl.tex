% This file was (origanally) converted to LaTeX by Writer2LaTeX ver. 1.4
% see http://writer2latex.sourceforge.net for more info
\documentclass[letterpaper]{article}
\include{data_parameters}
\usepackage[utf8]{inputenc}
\usepackage[T1]{fontenc}
\usepackage[french]{babel}
\usepackage{amsmath}
\usepackage{amssymb,amsfonts,textcomp}
\usepackage{color}
\usepackage{array}
\usepackage[hidelinks]{hyperref}
\usepackage{fancyhdr}
\usepackage{multirow}
\usepackage[flushleft]{threeparttable}
\usepackage{eurosym}

% Page layout (geometry)
\setlength\voffset{-1.5in}
\setlength\hoffset{-1in}
\setlength\topmargin{0.5in}
\setlength\oddsidemargin{1in}
\setlength\textheight{7.156in}
\setlength\textwidth{6.5in}
\setlength\footskip{0.5358in}
\setlength\headheight{1.1736in}
\setlength\headsep{0.5in}

% Pages styles
\pagestyle{fancy}
\fancyhf{}
\fancyhead[R,L]{\centering{\textbf{\companyName}}}
\fancyfoot[L]{\thepage}
\fancyfoot[R]{\DocTitle}

\fancypagestyle{firststyle} { \fancyhead[R,L]{\centering{ \companyHeader }}}

\author{}
\date{29-04-\GenYear}
\begin{document}

\setcounter{page}{1}
\title{ANNEXE\\Exercice \FacctYear}
\maketitle
\thispagestyle{firststyle}

\tableofcontents
\clearpage
\section{Résumé}
Ceci constitue l’annexe au bilan avant répartition de l'exercice, dont le total
 est de {\AnxTotalExercice} euros et au compte de résultat de l'exercice
 présenté sous forme de tableau, dont le chiffre d'affaires est de
 {\AnxChiffreAffaire} euros et dégageant {\AnxPerteBenefName} de
 {\AnxPerteBenef} euros.\par
Ce {\AnxNbYear} exercice a une durée de 12 mois, couvrant la période du
 01/01/{\FacctYear} au 31/12/{\FacctYear}.\par
Les notes et les tableaux présentés ci-après font partie intégrante des
 comptes annuels.

\section{Faits caractéristiques de l'exercice}
Il n’y a pas eu de fait significatif.

\section{Règles et méthodes comptables}
Code du commerce - articles L.123-12 et L.123-28\par
Décret n°83-1020 du 29/11/83\par
Règlement CRC n°99-133 :11CG\par
\subsection{Principes et conventions générales}
Les comptes de l'exercice clos ont été élaborés et présentés conformément aux
 règles comptables dans le respect des principes prévus par les articles 120-1
 et suivants du Plan Comptable General 2005.\\\par

Les conventions comptables ont été appliquées dans le respect du principe de
 prudence et conformément aux hypothèses de base suivantes:
\begin{itemize}
\item Continuité d’exploitation ;
\item Permanence de méthodes comptables ;
\item Indépendance des exercices ;
\end{itemize}

 et conformément aux règles générales d’établissement et de présentation des
 comptes annuels (les dispositions du code de commerce, du décret comptable du
 29/11/83 ainsi que des règlements CRC relatifs à la réécriture du plan
 comptable général 2005 applicables à la clôture de l'exercice).\\\par

La méthode de base retenue pour l’évaluation des éléments inscrits en
 comptabilité est la méthode des coûts historiques.\\\par

Le journal des écritures comptables est sauvegardé au format CSV pour des
 raisons d'interopérabilité logiciel et son exactitude est certifiée par
 l'indication de sa signature SHA1 enregistrée dans cet Annexe aux comptes
 (elle-même déposée au Tribunal de Commerce). La signature SHA1 pour les
 comptes {\FacctYear} est {\AnxShaKey}.
\subsection{Permanence des méthodes}
Aucun changement de méthode d'évaluation n’est intervenu au cours de ce premier
 exercice social.

\section{Complément d'informations relatif au bilan}
\subsection{État des immobilisations}
\begin{center}
   \begin{tabular}{
           |p{1.75in}
           |>{\centering\arraybackslash}p{1.04in}
           |>{\centering\arraybackslash}p{0.94in}
           |>{\centering\arraybackslash}p{0.94in}
           |>{\centering\arraybackslash}p{0.93in}| }
     \hline
     \multicolumn{5}{|c|}{\textbf{Tableau des immobilisations}}\\\hline
     Postes du bilan& Valeur brute au début de l'exercice& Augmentations&
       Diminutions& Valeur brute à la fin de l'exercice\\\hline
     Immobilisations incorporelles &
       {\AnxImmoIncorpBefore}&{\AnxImmoIncorpPlus}&
       {\AnxImmoIncorpMinus}&{\AnxImmoIncorpAfter} \\\hline
     Immobilisations corporelles &
       {\AnxImmoCorpBefore}&{\AnxImmoCorpPlus}&
       {\AnxImmoCorpMinus}&{\AnxImmoCorpAfter} \\\hline
     Immobilisations financières &
       {\AnxImmoFiBefore}&{\AnxImmoFiPlus}&
       {\AnxImmoFiMinus}&{\AnxImmoFiAfter} \\\hline
     \textbf{TOTAL} &
       {\AnxImmoTotalBefore}&{\AnxImmoTotalPlus}&
       {\AnxImmoTotalMinus}&{\AnxImmoTotalAfter} \\\hline
   \end{tabular}
 \end{center}
La Société n’a pas enregistré des immobilisations pendant l’exercice social
 {\FacctYear}.
\subsection{État des amortissements}
\begin{center}
   \begin{tabular}{
       |p{1.75in}
       |>{\centering\arraybackslash}p{1.04in}
       |>{\centering\arraybackslash}p{0.94in}
       |>{\centering\arraybackslash}p{0.94in}
       |>{\centering\arraybackslash}p{0.93in}| }
     \hline
     \multicolumn{5}{|c|}{\textbf{Tableau des amortissements}}\\\hline
     Postes du bilan& Cumuls au début de l'exercice& Augmentations& Diminutions&
     Cumuls à la fin de l'exercice\\ \hline
     Immobilisations incorporelles &
       {\AnxAmortIncorpBefore}&{\AnxAmortIncorpPlus}&
       {\AnxAmortIncorpMinus}&{\AnxAmortIncorpAfter} \\\hline
     Immobilisations corporelles &
       {\AnxAmortIncorpBefore}&{\AnxAmortIncorpPlus}&
       {\AnxAmortIncorpMinus}&{\AnxAmortIncorpAfter} \\\hline
     Immobilisations financières &
       {\AnxAmortIncorpBefore}&{\AnxAmortIncorpPlus}&
       {\AnxAmortIncorpMinus}&{\AnxAmortIncorpAfter} \\\hline
     \textbf{TOTAL} &
       {\AnxAmortIncorpBefore}&{\AnxAmortIncorpPlus}&
       {\AnxAmortIncorpMinus}&{\AnxAmortIncorpAfter} \\\hline
   \end{tabular}
 \end{center}
La Société n’a pas effectué des amortissements sur des immobilisations, étant
 donné qu’elle n’a pas enregistré des immobilisations pendant l’exercice social
 {\FacctYear}.
\subsection{État des provisions}
\begin{center}
   \begin{tabular}{
       |p{1.75in}
       |>{\centering\arraybackslash}p{1.04in}
       |>{\centering\arraybackslash}p{0.94in}
       |>{\centering\arraybackslash}p{0.94in}
       |>{\centering\arraybackslash}p{0.93in}| }
     \hline
     \multicolumn{5}{|c|}{\textbf{Tableau des provisions}}\\\hline
     Postes du bilan& Cumuls au début de l'exercice& Augmentations& Diminutions&
     Cumuls à la fin de l'exercice\\ \hline
     Provisions réglementées &
       {\AnxProvRegleBefore}&{\AnxProvReglePlus}&
       {\AnxProvRegleMinus}&{\AnxProvRegleAfter} \\\hline
     Provisions pour risques &
       {\AnxProvRiskBefore}&{\AnxProvRiskPlus}&
       {\AnxProvRiskMinus}&{\AnxProvRiskAfter} \\\hline
     Provisions pour charges &
       {\AnxProvChargeBefore}&{\AnxProvChargePlus}&
       {\AnxProvChargeMinus}&{\AnxProvChargeAfter} \\\hline
     Provisions pour dépréciations &
       {\AnxProvDepreciaBefore}&{\AnxProvDepreciaPlus}&
       {\AnxProvDepreciaMinus}&{\AnxProvDepreciaAfter} \\\hline
     \textbf{TOTAL} &
       {\AnxProvTotalBefore}&{\AnxProvTotalPlus}&
       {\AnxProvTotalMinus}&{\AnxProvTotalAfter} \\\hline
   \end{tabular}
 \end{center}
La Société a constitué des provisions pour charges sociales pour l’exercice
 fiscal {\FacctYear}.
\subsection{État des échéances, des créances et des dettes}
\begin{center}
  \begin{threeparttable}
   \begin{tabular}{
       |p{3.08in}
       |>{\centering\arraybackslash}p{0.8in}
       |>{\centering\arraybackslash}p{0.95in}
       |>{\centering\arraybackslash}p{0.95in}| }
     \hline
     \multirow{2}{*}{\textbf{Créances (a)}} &
     \multirow{2}{*}{Montant brut}&
     \multicolumn{2}{c|}{Liquidité de l'actif}\\\cline{3-4}
      &  &A plus d'un an&A plus de 5 ans\\\hline
     \multicolumn{4}{|l|}{\textbf{Créances de l'actif immobilisé :}}
       \\\hline
     \ - Créances rattachées à des participations &
       {\AnxCreancesAttachedGross}&{\AnxCreancesAttachedOne}&
       {\AnxCreancesAttachedFive} \\\hline
     \ - Prêts (1) &
       {\AnxCreancesLoansGross}&{\AnxCreancesLoansOne}&
       {\AnxCreancesLoansFive} \\\hline
     \ - Autres &
       {\AnxCreancesImmoOtherGross}&{\AnxCreancesImmoOtherOne}&
       {\AnxCreancesImmoOtherFive} \\\hline
     \multicolumn{4}{|l|}{\textbf{Créances de l'actif circulant :}}
       \\\hline
     \ - Créances Clients et Comptes rattachés &
       {\AnxCreancesClientsGross}&{\AnxCreancesClientsOne}&
       {\AnxCreancesClientsFive} \\\hline
     \ - Autres &
       {\AnxCreancesOtherCurrentGross}&{\AnxCreancesOtherCurrentOne}&
       {\AnxCreancesOtherCurrentFive} \\\hline
     \ - Capital souscrit - appelé, non versé &
       {\AnxCreancesCapitalGross}&{\AnxCreancesCapitalOne}&
       {\AnxCreancesCapitalFive} \\\hline
     \ - Charges constatées d'avance &
       {\AnxCreancesAdvanceGross}&{\AnxCreancesAdvanceOne}&
       {\AnxCreancesAdvanceFive} \\\hline
     \textbf{TOTAL} &
       {\AnxCreancesTotalGross}&{\AnxCreancesTotalOne}&
       {\AnxCreancesTotalFive} \\\hline
   \end{tabular}
   \begin{tablenotes}
      \scriptsize
      \item {(1) Prêts accordés ou récupérés en cours d'exercice}
    \end{tablenotes}
  \end{threeparttable}
 \end{center}

\begin{center}
  \begin{threeparttable}
   \begin{tabular}{
           |p{2.6in}
           |>{\centering\arraybackslash}p{0.8in}
           |>{\centering\arraybackslash}p{0.8in}
           |>{\centering\arraybackslash}p{0.7in}
           |>{\centering\arraybackslash}p{0.7in}| }
     \hline
     \multirow{2}{*}{\textbf{Dettes (b)}} &
     \multirow{2}{*}{Montant brut}&
     \multirow{2}{0.7in}{Échéances à moins d'un an}
     &\multicolumn{2}{c|}{Échéances}\\\cline{4-5}
      & & &A plus d'un an&A plus de 5 ans\\\hline
     Emprunts obligataires convertibles (2) &
       {\AnxDettesBorrowConvGross}&{\AnxDettesBorrowConvMinus}&
       {\AnxDettesBorrowConvOne}&{\AnxDettesBorrowConvFive}\\\hline
     Autres emprunts obligataires (2) &
       {\AnxDettesBorrowOtherGross}&{\AnxDettesBorrowOtherMinus}&
       {\AnxDettesBorrowOtherOne}&{\AnxDettesBorrowOtherFive}\\\hline
     \multicolumn{5}{|l|}{
         Emprunts (2) et dettes auprès des établissements de crédit dont:}
     \\\hline
      \ - à 2 ans au maximum à l'origine &
       {\AnxDettesTwoMaxGross}&{\AnxDettesTwoMaxMinus}&
       {\AnxDettesTwoMaxOne}&{\AnxDettesTwoMaxFive}\\\hline
      \ - à plus de 2 ans à l'origine &
       {\AnxDettesOverTwoGross}&{\AnxDettesOverTwoMinus}&
       {\AnxDettesOverTwoOne}&{\AnxDettesOverTwoFive}\\\hline
     Emprunts et dettes financières divers (2) (3) &
       {\AnxDettesMiscaGross}&{\AnxDettesMiscaMinus}&
       {\AnxDettesMiscaOne}&{\AnxDettesMiscaFive}\\\hline
     Dettes fournisseurs et comptes rattachés &
       {\AnxDettesSuppliersGross}&{\AnxDettesSuppliersMinus}&
       {\AnxDettesSuppliersOne}&{\AnxDettesSuppliersFive}\\\hline
     Dettes fiscales et sociales &
       {\AnxDettesTaxesGross}&{\AnxDettesTaxesMinus}&
       {\AnxDettesTaxesOne}&{\AnxDettesTaxesFive}\\\hline
     Dettes sur immobilisations et comptes rattachés &
       {\AnxDettesImmoGross}&{\AnxDettesImmoMinus}&
       {\AnxDettesImmoOne}&{\AnxDettesImmoFive}\\\hline
     Autres dettes (3) &
       {\AnxDettesOtherGross}&{\AnxDettesOtherMinus}&
       {\AnxDettesOtherOne}&{\AnxDettesOtherFive}\\\hline
     Produites constatés d'avance &
       {\AnxDettesAdvIncomeGross}&{\AnxDettesAdvIncomeMinus}&
       {\AnxDettesAdvIncomeOne}&{\AnxDettesAdvIncomeFive}\\\hline
     \textbf{TOTAL} &
       {\AnxDettesTotalGross}&{\AnxDettesTotalMinus}&
       {\AnxDettesTotalOne}&{\AnxDettesTotalFive}\\\hline
   \end{tabular}
   \begin{tablenotes}
      \scriptsize
      \item {(2) Emprunts souscrits ou remboursés en cours d'exercice}
	  \item{(3) Dont un montant de 0\euro\ envers les associés}
    \end{tablenotes}
  \end{threeparttable}
 \end{center}
Les dettes fiscales sont constituées par des montants estimés dus à
 l’Administration Fiscale au titre de l’impôt sur les sociétés, étant donné que
 la Société a fait un bénéfice, et au titre de la taxe professionnelle pour
 l’exercice social {\FacctYear}. Les dettes sociales sont constituées des
 cotisations estimées dues aux organismes sociaux au titre du salaire du Gérant
 pour l’exercice social {\FacctYear}.
\subsection{Évaluation des amortissements}
Néant/Non applicable
\subsection{Évaluation des créances et des dettes}
{\scriptsize Décret n°83-1020 du 29/11/83 article 24- 5°}\\\par
Les créances et dettes ont été évaluées pour leur valeur nominale.
\subsection{Charges à payer}
{\scriptsize Décret no. 83-1020 du 29/11/83 article 23}\\\par
\begin{center}
   \begin{tabular*}{\textwidth}{ |l @{\extracolsep{\fill}}| c | }
     \hline
     Montant des charges à payer incluses dans les postes suivants du bilan &
       Montant \\\hline
     Charges constatées d’avance & {\AnxCToPayBefore} \\\hline
     Total & {\AnxCToPayTotal} \\
     \hline
   \end{tabular*}
 \end{center}
La Société {\AnxCToPayHaveOrNot} enregistré une charge constatée d’avance.

\section{Engagements financiers et autres informations}
La Société n’a donné aucun engagement pendant l’exercice {\FacctYear}.
\end{document}
